\section{Hardware and operating system environments} % (fold)
\label{sec:hardware_and_operating_system_environments}

The main calculations (for Table~~3, for instance) were carried out on a Wang 2200B, a high-end business-oriented microcomputer.  A dialect of BASIC, very close to the original language described by Kemeny and Kurtz, and Keller in 1964\footnote{In fact, the only available variable names were the single letters A--Z and a letter followed by a single digit.  Array names had the same limitation, but the array and scalar namespaces were distinct.   Thus, one could have both a scalar variable \texttt{W} and an array named \texttt{W}, leading to constructions like \texttt{MAT W=(1/W)*W}, a line that actually appears in the code used to generate Table~3.} was embedded in its microcode.  It could only be programmed in BASIC.  Programs had to be entered at the keyboard, but could be stored and retrieved using a cassette tape recorder. Programs could be stored for posterity (and future reproducibility studies), provided posterity was not expected to last longer than two or three years.
s
The programs written in PL/I (a general-purpose programming language of the day) and the ones written as input to MPS/360 were run on an IBM/360 mainframe computer.  The days of using punched cards for such programs were waning, and it had just become possible to create, save, and edit text files interactively, and then to submit them as if they were stacks of punched cards for execution.  I do not recall whether I carted boxes of punched computer cards with me when I left Stanford for Chicago in 1976, but I can say definitively that none remain.

One of the challenges for reproducible research is that programming languages and software systems (such as Stata, R, or MPS/360) can either become obsolete or evolve to the point of unrecognizability (and incompatibility).  The hardware on which programs run also becomes obsolete or unavailable.  And even were the same hardware still available, the underlying programs may no longer run under current versions of the hardware's operating system.

There is one additional aspect of computer hardware that can affect reproducibility, especially in the context of the Shakespeare example.  In 1975, numerics were poorly understood by computer architects of the day.  The first IEEE Standards for floating-point computation did not exist until 1985, ten years after our work.  As a result, the numerical quality of computed results could be quite poor, and without there being any indicator that that was the case.  So even if the calculations could be reproduced on the hardware of the day, there would still be good reason to check to see whether the results are different today!

% section hardware_and_operating_system_environments (end)
